Este trabajo se enfoca en el modelamiento y descubrimiento de tópicos en el tiempo en un corpus. El enfoque descrito comienza con la discretización del corpus en épocas. Luego, usando como aproximación que la estructura de los tópicos dentro de cada época es estática, los tópicos son descubiertos usando Hierarchical Dirichlet Process. Por último, la evolución de los tópicos en el tiempo es modelada por un grafo temporal sustentado por una medida de similitud entre tópicos. El grafo inicialmente es construido por los arcos entre todos los pares de tópicos de épocas adyacentes, luego es podado automáticamente en base a un punto operante de la cdf de la similitud. Esta estructura permite inferir cambios estructurales complejos de los tópics, como nacimiento, muerte, división, fusión y evolución en el tiempo.\\

En contraste a trabajos anteriores, la metodología propuesta utiliza Word Mover's Distance como medida de similitud entre tópicas, permitiendo comparar de forma más apropiada tópicos que no poseen un vocabulario común a través de sus \textit{word embeddings}. Además, se presenta una análisis empírico del \texit{trade off} entre precisión y \textit{speedup} de no utilizar el vocabulario completo del tópico en la construcción del grafo de similitud.\\

Resultados experimentales al fenómeno de robo de vehículos son reportados. Se muestra el efecto que tiene la elección del punto operante de la cdf en la construcción del grafo final. Otro importante hallazgo, es nivel de consistencia de la metodología en modelar de la evolución de los tópicos en el tiempo, siendo capaz de relacionar tópicos que evidentemente son similares, como es el caso del robo no presencial o el robo con violencia.\\

La metodología propuesta puede tener otros usos interesantes como descubrir nuevas tendencias de investigación, analizar la evolución de la contigencia social, estudiar la efectividad de campañas publicitarias en base a la opinión de los consumidores, organizar y recomendar contenido en un blog, etc.\\

Como trabajo futuro podría ser interesante extender la metodología a un enfoque puramente basado en redes neuronales. De esta forma la comparación entre tópicos de épocas adyacentes a través de sus \textit{word embeddings} se vuelve más natural. También sería más consistente ya que la información codifica en los \textit{word embeddings} se utilizaría para el mismo descubrimiento de los tópicos. En \cite{dieng2019dynamic} se propone una modelo de tópicos dinámico basado en redes neuronales, sin embargo, este mantiene fijo el número de tópicos en el tiempo.
